\documentclass[10pt,prd,nofootinbib,superscriptaddress,twocolumn]{revtex4-2}
\pdfoutput=1


%%%%%%%%%%%%% Packages %%%%%%%%%%%%%
% general
\usepackage[utf8]{inputenc}
\usepackage{geometry}
\geometry{
	a4paper,
	total={170mm,257mm},
	left=20mm,
	top=20mm,
}

% math
\usepackage{mathtools}
\usepackage{amsfonts}
\usepackage{dsfont}
\usepackage{mathrsfs}
\usepackage{bbm}
\usepackage[normalem]{ulem}
\usepackage{slashed}
\usepackage{tensor}

\usepackage{pifont}
\newcommand{\xmark}{\ding{55}}

% graphics and colours
\usepackage{graphicx}
\usepackage{array}

% floats
\usepackage{placeins}
\usepackage{makecell}
\usepackage{float}

% units and refs
\usepackage{xspace}
\usepackage{xfrac}
\usepackage{hyperref}
\usepackage[nameinlink]{cleveref}
\usepackage{appendix}
\usepackage{units}


% other
\usepackage{xifthen}
\usepackage{booktabs}
\usepackage{multirow}
\usepackage[dvipsnames]{xcolor}
\hypersetup{
	colorlinks,
	linkcolor={red!75!black},
	citecolor={blue!75!black},
	urlcolor={blue!75!black},
	pdftitle={Juggling with Tensor Bases in Functional Approaches},
	pdfauthor={Braun, GeissŸel, Pawlowski, Sattler, Wink},
}
\usepackage{enumerate}


\newcommand{\innerproduct}[2]{\left\langle #1, #2 \right\rangle}
\newcommand{\imag}{\text{i}}
\newcommand{\norm}[1]{\left\lVert#1\right\rVert}

%%%%%%%%%%%%% Comments %%%%%%%%%%%%%
\newcommand{\commentFS}[1]{\textcolor{blue}{[\textbf{FS:} #1]}}
\newcommand{\commentNW}[1]{\textcolor{orange}{[\textbf{NW:} #1]}}
\newcommand{\commentAG}[1]{\textcolor{purple}{[\textbf{AG:} #1]}}
\newcommand{\cJMP}[1]{\textcolor{blue}{\textbf{Jan: #1}}}

%%%%%%%%%%%%% Math commands %%%%%%%%%%%%%
% symbols
\newcommand{\tinytext}[1]{\text{\tiny{#1}}}
\newcommand{\TensorBases}{\texttt{TensorBases}\xspace}

%%%%%%%%%%%%% Inline code shortcuts %%%%%%%%%%%%%
\newcommand{\mathem}{\mmaInlineCell{Code}}
\newcommand{\mathcell}[1]{
	\begin{mmaCell}{Input}
		#1
	\end{mmaCell}
}

%%%%%%%%%%%%% Mathematica code %%%%%%%%%%%%%
\usepackage{mmacells}
\mmaDefineMathReplacement[≤]{<=}{\leq}
\mmaDefineMathReplacement[≥]{>=}{\geq}
\mmaDefineMathReplacement[≠]{!=}{\neq}
\mmaDefineMathReplacement[→]{->}{\to}[2]
\mmaDefineMathReplacement[⧴]{:>}{:\hspace{-.2em}\to}[2]
\mmaDefineMathReplacement{∉}{\notin}
\mmaDefineMathReplacement{∞}{\infty}
\mmaDefineMathReplacement{𝕕}{\mathbbm{d}}
\mmaSet{
	leftmargin=2.3em,
	morefv={gobble=0},
	linklocaluri=mma/symbol/definition:#1,
	morecellgraphics={yoffset=1.9ex},
	labelsep=0.1em
}

%%%%%%%%%%%%%%%%%%% MARCROS %%%%%%%%%%%%%
\newcommand{\LEGO}{LEGO\textsuperscript{\textregistered}}
\newcommand{\DiFfRG}{\texttt{DiFfRG}\xspace}

%%%%%%%%%%%%% Graphic paths %%%%%%%%%%%%%
\graphicspath{{./figures/}}

\renewcommand{\floatpagefraction}{.8}


%%%%%%%%%%%%% Title and hypersetup %%%%%%%%%%%%%
\newcommand{\gettitle}{FunKit: A toolkit for functional approaches}

\newcommand{\getHeidelbergAffiliation}{\affiliation{Institut f{\"u}r Theoretische Physik, Universit{\"a}t Heidelberg, Philosophenweg 16, 69120 Heidelberg, Germany}}
\newcommand{\getDarmstadtAffiliation}{\affiliation{Institut f\"ur Kernphysik (Theoriezentrum), Technische Universit\"at Darmstadt, 64289 Darmstadt, Germany}}
\newcommand{\getEMMIAffiliation}{\affiliation{ExtreMe Matter Institute EMMI, GSI, Planckstr. 1, 64291 Darmstadt, Germany}}


\begin{document}

\title{\gettitle}

\begin{abstract}

We give tool to compute cross-section
\end{abstract}

\maketitle


%%%%%%%%%%%%%%%%%%%%%%%%%%%%%%
\section{FED\MakeLowercase{eri}K}

abc


%%%%%%%%%%%%%%%%%%%%
\subsection{Rules}
\label{sec:rules}

\subsection{Index \& derivative rules}
\label{sec:rules_indicesDerivatives}

Metric:
%
\begin{align}
	\gamma^{ab}=\gamma_{ab} = \begin{pmatrix}
		0&-1\\
		1&\phantom{-}0
	\end{pmatrix}
	\,.
\end{align}
%
Metric contractions:
%
\begin{align}
	\gamma^{a}_{\phantom{a}c} &= \gamma^{ab}\gamma_{bc} = (-1)^{ac}\delta^a_c
	\,,\notag\\[1ex]
	\gamma_{a}^{\phantom{a}c} &= \gamma^{bc}\gamma_{ba} = \delta^a_c
	\,.
\end{align}
%
As a consequence, $\gamma^{a}_{\phantom{a}c}\gamma^{c}_{\phantom{c}d} = \delta^{ad}$.

Field derivatives:
%
\begin{align}
	\frac{\delta \Phi^b}{\delta \Phi^a} &= \gamma_a^{\phantom{a}b}
	\,,\qquad
	\frac{\delta \Phi_b}{\delta \Phi^a} = \gamma_{ab}
	\,,\notag\\[1ex]
	\frac{\delta \Phi_b}{\delta \Phi_a} &= \gamma^{ba}
	\,,\qquad
	\frac{\delta \Phi_b}{\delta \Phi_a} = \gamma^{a}_{\phantom{a}b}
	\,.
\end{align}
%
N-point functions:
%
\begin{align}
	\frac{\delta }{\delta \Phi^a} \Gamma_{bc\ldots} = \Gamma_{abc\ldots}
	\,,\notag\\[1ex]
	\frac{\delta }{\delta \Phi_a} \Gamma_{bc\ldots} = \Gamma^{a}_{\phantom{a}bc\ldots}
	\,.
\end{align}
%
Propagator inverse:
%
\begin{align}
	\frac{\delta \Gamma}{\delta \Phi^a} &= J_a
	\,,\notag\\[1ex]\Leftrightarrow 
	\frac{\delta \Gamma}{\delta J_b\delta \Phi^a} &= \delta_a^{\phantom{a}b} = \gamma_a^{\phantom{a}b}
	\,,\notag\\[1ex]\Leftrightarrow 
	\frac{\delta \Phi^c}{\delta J_b}\frac{\delta \Gamma}{\delta \Phi^c\delta \Phi^a} &= \gamma_a^{\phantom{a}b}
	\,,\notag\\[1ex]\Leftrightarrow 
	G^{bc}\Gamma_{ca} &= \gamma_a^{\phantom{a}b}
	\,.
\end{align}
%
and
%
\begin{align}
	\frac{\delta \Gamma}{\delta \Phi^a} &= J_a
	\,,\notag\\[1ex]\Leftrightarrow 
	\frac{\delta \Gamma}{\delta \Phi^a\delta J_b} &= \gamma^b_{\phantom{b}a}
	\,,\notag\\[1ex]\Leftrightarrow 
	\frac{\delta \Gamma}{\delta \Phi^a\delta \Phi^c}\frac{\delta \Phi^c}{\delta J_b} &= \gamma^b_{\phantom{b}a}
	\,,\notag\\[1ex]\Leftrightarrow 
	\Gamma_{ac}G^{bc} &= \gamma^b_{\phantom{b}a}
	\,.
\end{align}
%
%
\begin{align}
	\left(\frac{\delta}{\delta \Phi^f}G^{ba}\right)= (-1)\gamma^a_{\phantom{a}e}\, G^{bc}\Gamma_{fcd}G^{ed}
	\,.
\end{align}
%
Getting the rule for the propagator:
%
\begin{align}
	\left(\frac{\delta}{\delta \Phi^f}G^{bc}\right)\Gamma_{cd} &= (-1)(-1)^{bf}\,G^{bc}\Gamma_{cfd}
	\,,\notag\\[1ex]
	\Leftrightarrow\left(\frac{\delta}{\delta \Phi^f}G^{bc}\right)(\Gamma_{cd}G^{ed}) &= (-1)(-1)^{bf}\,G^{bc}\Gamma_{cfd}G^{ed}
	\,,\notag\\[1ex]
	\Leftrightarrow\left(\frac{\delta}{\delta \Phi^f}G^{bc}\right)\gamma^e_{\phantom{e}c} &= (-1)(-1)^{bf}\,G^{bc}\Gamma_{cfd}G^{ed}
	\,,\notag\\[1ex]
	\Leftrightarrow\left(\frac{\delta}{\delta \Phi^f}G^{ba}\right) &= (-1)(-1)^{bf}\gamma^a_{\phantom{a}e}\, G^{bc}\Gamma_{cfd}G^{ed}
\end{align}
%%%%%%%%%%%%%%%%%%%%

\end{document}